\documentclass{article}
\usepackage[a4paper, total={6in, 8in}]{geometry}

\usepackage{listings}
\usepackage{color}

\definecolor{mygreen}{rgb}{0,0.6,0}
\definecolor{mygray}{rgb}{0.5,0.5,0.5}
\definecolor{mymauve}{rgb}{0.58,0,0.82}

\lstset{ %
   backgroundcolor=\color{white},   % choose the background color; you must add \usepackage{color} or \usepackage{xcolor}
   basicstyle=\footnotesize,        % the size of the fonts that are used for the code
   breakatwhitespace=false,         % sets if automatic breaks should only happen at whitespace
   breaklines=true,                 % sets automatic line breaking
%   captionpos=b,                    % sets the caption-position to bottom
   commentstyle=\color{mygreen},    % comment style
   deletekeywords={...},            % if you want to delete keywords from the given language
   escapeinside={\%*}{*)},          % if you want to add LaTeX within your code
   extendedchars=true,              % lets you use non-ASCII characters; for 8-bits encodings only, does not work with UTF-8
   frame=single,                    % adds a frame around the code
   keepspaces=true,                 % keeps spaces in text, useful for keeping indentation of code (possibly needs columns=flexible)
   keywordstyle=\color{blue},       % keyword style
   language=C,                 % the language of the code
   morekeywords={*,...},            % if you want to add more keywords to the set
   numbers=left,                    % where to put the line-numbers; possible values are (none, left, right)
   numbersep=5pt,                   % how far the line-numbers are from the code
   numberstyle=\tiny\color{mygray}, % the style that is used for the line-numbers
   rulecolor=\color{black},         % if not set, the frame-color may be changed on line-breaks within not-black text (e.g. comments (green here))
   showspaces=false,                % show spaces everywhere adding particular underscores; it overrides 'showstringspaces'
   showstringspaces=false,          % underline spaces within strings only
   showtabs=false,                  % show tabs within strings adding particular underscores
   stepnumber=1,                    % the step between two line-numbers. If it's 1, each line will be numbered
   stringstyle=\color{mymauve},     % string literal style
   tabsize=2,                       % sets default tabsize to 2 spaces
%   title=\lstname                   % show the filename of files included with \lstinputlisting; also try caption instead of title
% in the end language and linenumbres will be called for each block, otherwise html conversion doesn't pick up on the proper markup ...
}
% tweak the the style a little bit
\lstdefinestyle{customc}{
%   belowcaptionskip=1\baselineskip,
   breaklines=true,
   frame=single,
   xleftmargin=\parindent,
   language=C,
%   showstringspaces=false,
   basicstyle=\footnotesize\ttfamily,
   keywordstyle=\bfseries\color{green!40!black},
   commentstyle=\itshape\color{mygray},
   identifierstyle=\color{blue},
   stringstyle=\color{orange},
   framexleftmargin=5mm,
}

\lstset{escapechar=@,style=customc}
\usepackage[sort&compress,numbers]{natbib}


\usepackage{graphicx}
\usepackage{caption}
\usepackage{subcaption}
\usepackage{hyperref}

\title{ALICE tutorial session}
\author{ Redmer Alexander Bertens - rbertens@cern.ch \\ Mike Sas - mike.sas@cern.ch }
\date{\today}

\setlength\parindent{0pt}

\begin{document}

\maketitle

\section{Run the example}

Our starting point for these exercises is the AnalysisTask: "AliAnalysisTaskMyTask.cxx" that was covered during the presentation. To run this task, you need to
\begin{itemize}
    \item download an input file \textbf{AliAOD.root}, which contains real events from the 2010 Pb--Pb run. You can download it at
\href{https://cernbox.cern.ch/index.php/s/ZP2gJBE265FiSAX}{https://cernbox.cern.ch/index.php/s/ZP2gJBE265FiSAX} , the password is `croissant' (we are in France). 
\item download a .tar file with the full code from the tutorial webpage \href{https://indico.cern.ch/event/647269/}{https://indico.cern.ch/event/647269/}
    \end{itemize}
    Store these files on your laptop. \emph{The path in which you store these files cannot contain spaces! E.g. /home/my awesome task/ will not work, use /home/my\_awesome\_task/}


\subsection{Testing the task}

As a start, just take a look at code that makes up the task. First, create a folder in which you want to run your analysis, and extract the contents of the .tar file to this folder. 


When you've extracted the .tar file, you should see the files that were also covered in the first part of the talk:

\begin{itemize}
\item AliAnalysisTask.cxx
\item AliAnalysisTask.h
\item AddMyTask.C
\item runAnalysis.C
\end{itemize}

try to see where all the code snippets that were shown during the presentation fit in. 

See if you can answer the following questions
\begin{itemize}
\item which function is called for each event?
\item where are the output histogram defined?
\item where are the output histogram filled?
\end{itemize}

\subsection{Running the task}
It's time to run this simple analysis. If you don't have AliROOT running on your laptop already, ask for our help and/or team up with one of your fellow summer students. To run the task, open a terminal, source your aliroot environment (think of the `alienv' commands that Dario explained about yesterday) and type

\$ aliroot runAnalysis.C\\

This launches aliroot, and executes the macro 'runAnalysis.C'. Make sure though, that the macro 'knows' where you stored the AliAOD.root file on your laptop (if the AliAOD.root file is in a different folder than the analysis source code, open the runAnalysis.C file, and change the lines where the input file is accessed). 


\subsection{Take a look at the output}
If all goes well, your simple task runs over the input data, and fills one histogram with the distribution of transverse momentum ($p_{\rm T}$) of all charged particle tracks in all events. Once the task is done running, take a look at it. Open a TBrowser (ROOT's 'graphical user interface') and take a look at the distribution in the histogram. No idea how to open a TBrowser, or what a TBrowser even is? Don't hesitate to ask, and take a look at the \href{https://root.cern.ch/root/htmldoc/guides/users-guide/ROOTUsersGuide.html}{ROOT user's guide}.

\section{Event properties}

If you managed to run the task, congratulations! This is step one towards becoming a professional physicist (no turning back from here ... ) .   In this exercise we will add code to our analysis class to obtain more information about the collisions. First, create additional output histograms in which you'll store
\begin{itemize}
\item the total number of events, named "NEvents". This will be extended later is this exercise
\item the primary vertex positions along the beam line for all events, named "VtxPos"
\end{itemize}

Try to run the code again, after you've added these histograms. 

\subsection{All set?}

Not all collisions in the detector are actually usable for physics analysis. Events which happen very far from the center of the detector, for example, are usually not used for analysis (any idea why?). To make sure that we only look at high quality data, we do 'event selection'. Let's add some event selection criteria to your task. 
\begin{itemize}
\item modify the code such that only primary vertices are accepted within 10 cm of detector center; $|z_{vtx}|<10$cm
\item modify NEvents to store the information on how many events are accepted and declined due to the vertex cut.Tip: use a histogram as a simple 'bookkeeping' device, fill it with a 0 if an event is rejected, and a 1 when the event is accepted. More creative solutions are always welcome!
\end{itemize}


\section{Average event multiplicity and track selection}

In this exercise we are going to study the particles that are produced in each event. First:
\begin{itemize}
\item plot the $\eta$ and $\varphi$ distributions of the particles for all accepted events in a two-dimensional histogram
\item make sure that the axis have proper ranges (how large is the detector?) and a reasonable amount of bins
\end{itemize}

If you've added this historam, run the analysis, and look at the two dimensional histogram. Does the distribution make sense?\\

Just as some events are not of high enough quality to do analysis with, some tracks are also not good for looking at certain observables. In addition to 'event cuts', we therefore have to apply 'track cuts' as well. In ALICE, we use a system of predefined track selection criteria, that are checked by reading the 'filterbit' of a track. 

\begin{itemize}
    \item Check in the code where the filterbit selection is done
    \item Change the filterbit selection from value 128 to 512
\end{itemize}
How does your $\eta$ $\phi$ distribution change, if you run your task again with filterbit 512? What do you think is the cause of the difference? Please ask one of us to explain this in a bit more detail. 



\section{Particle identification using the TPC}

Apparently you are very fast, because you've already reached the second part of the exercises. In this section, we'll try to identify particles, by using the amount of energy they have lost per distance traveled in the Time Projection Chamber (TPC). Some specific technical information on how to implement this in your task is given at the end of this document, but general steps to follow are explained here. First, the goals. What we want to do is
\begin{itemize}
\item identify pions
\item check how 'pure' our pion sample is
\end{itemize}

\includegraphics[width=\textwidth]{pid.png}


We'll do this by looking at the TPC signal, and seeing how well this signal corresponds to the expected signal for a certain particle species. Look at the figure for clarification, you see lines (the expected signal a particle leaves in the TPC, `the hypothesis') and in colors, the actual signals that were deposited.


\\


Start by storing the TPC signals for all charged particles in your events, in a two-dimensional histogram (such as shown in the figure). First follow the 'technical steps' from Sec. 8, and then try to make such a plot. 

\\


Hint, you can get this information by doing
\begin{lstlisting}[language=C,number=left]
fYour2DHistogram->Fill(track->P(), track->GetTPCsignal());\end{lstlisting}
If your histogram is \emph{empty} after running, try using Filterbit 1 rather than 128.

\\ 


As a second step, you identify particles, and store only the TPC signal that corresponds to pions. Use the 'NumberOfSigmasTPC' functions explained in Sec. 8. 
\\


If you are confident that you've 'isolated' the pions, create new histograms to store the pion's
\begin{itemize}
\item raw $p_{T}$ spectrum
\item $\eta$
\item $\varphi$
\end{itemize}

\section{Minimal systematics}

As you can see in the figure, the TPC signals of different particle species cross eachother. In these regions, it is difficult (or impossible) to identify particles. Because of this ambiguity, we have to estimate what the uncertainty on our identification routine is. 

\\


\begin{itemize}
\item run the main code multiple times, varying the number of standard deviations that you require the signal to be in from 3 to 4, and from 3 to 2
\item how do your spectra change? do these changes make sense to you?
\item if you are very enthousiastic, try to estimate the \emph{systematic uncertainty} that particle identification introduces to your measurement
\end{itemize}


\section{Make it presentable}

By now, you have made quite a few nice plots. Modify the code such that the labels, titles, markers, colors, legends of your plots look decent and presentable for fellow colleagues
\begin{itemize}
\item Did you know that you can use \LaTeX in root? this comes in handy when you are labeling axes, etc
\item Store one or two of your pictures. You can store them as ROOT files, macros, or as graphics. Always use vector graphics (.ps, .pdf)! 
\item Never be afraid to make titles and axis labels larger, it's better if things look a bit 'cartoonish' than when half of the collaboration cannot read what's on your plots - especially when they are broadcast or projected
\end{itemize}

In practice, we usually don't use the TBrowser for 'postprocessing', but write dedicated code that opens our AnalysisResults.root files, does some final calculations, and makes our plot pretty. 
\section{Some hints}

To create documentation of analysis code, we use a system called 'Doxygen'. You can find explanation about Doxygen at this webpage: \href{https://dberzano.github.io/alice/doxygen/}{https://dberzano.github.io/alice/doxygen/}. 
\\

The Doxygen documentation of AliROOT and AliPhysics is available here \href{http://alidoc.cern.ch/}{http://alidoc.cern.ch/} . It is generated daily. 

\\

\section{PID response}
To speed things up a bit, the code that you need for excercises 4 and onwards is given here below. 

\\
To identify particles, you will add an additional task to your analysis, the `PID response' task. This task makes sure that parametrizations that we use for e.g. specific energy loss in the TPC are loaded. To add this task to your analysis, open your runAnalysis.C macro, and add the following lines. Make sure that these lines are called \emph{before} your own task is added to that analysis manager, your task will \emph{depend} on this task:

\begin{lstlisting}[language=C, number=left]
   // load the necessary macros
   gROOT->LoadMacro("$ALICE_ROOT/ANALYSIS/macros/AddTaskPIDResponse.C");
   AddTaskPIDResponse();\end{lstlisting}


As a second step, you'll have to make some changed to the header of our analysis. First, you'll add another class member, which is a pointer to the AliPIDresponse object that we'll use in our class. To do this, remember how you added pointers to histograms earlier in this tutorial, so add
\begin{itemize}
\item a \emph{forward declaration} of the type, i.e. `class   AliPIDResponse'
\item the pointer itself, i.e.   `AliPIDResponse*       fPIDResponse;   //! pid response object'
\end{itemize}
   Don't forget to add this new pointer fPIDResponse to the member initialization list in your class constructors (in the .cxx files)!
\\

After making changes to the header, you can add the actual identification routine. This will be done in the implementation of the class. You need to go through a few steps:

\begin{itemize}
\item Retrieve the AliPIDResponse object from the analysis manager;
\item make our poiter `fPIDResponse' point to the AliPIDResponse object;
\item include the header for the AliPIDResponse class, so that the compiler knows how it is defined
\end{itemize}

The code to do this might be a bit difficult to figure out by yourself, so it's given here below
\begin{lstlisting}[language=C, number=left]
        AliAnalysisManager *man = AliAnalysisManager::GetAnalysisManager();
        if (man) {
            AliInputEventHandler* inputHandler = (AliInputEventHandler*)(man->GetInputEventHandler());
            if (inputHandler)   fPIDResponse = inputHandler->GetPIDResponse();
        }\end{lstlisting}
Do you understand
\begin{itemize}
\item what happens in each line?
\item in which function (UserCreateOutputObjects, UserExcec, Terminate, etc) to put this code?
\end{itemize}

Now that we've retrieved our PID response object, it's time to use it. To extract information for different particle species, you can use the functions
\begin{lstlisting}[language=C, number=left]
    Double_t kaonSignal = fPIDResponse->NumberOfSigmasTPC(track, AliPID::kKaon);
    Double_t pionSignal = fPIDResponse->NumberOfSigmasTPC(track, AliPID::kPion);
    Double_t protonSignal = fPIDResponse->NumberOfSigmasTPC(track, AliPID::kProton);\end{lstlisting}

What these functions return you, is a probability that a particle is of a certain type, expressed in standard deviations ($\sigma$) . To identify a particle, we'll require that its PIDresponse signal lies \emph{withim 3 standard deviations} of the expected signal. To check this, you can write a few lines such as
\begin{lstlisting}[language=C, number=left]
      if(TMath::Abs(fPIDResponse->NumberOfSigmasTPC(track, AliPID::kPion)) < 3 ) {
            // jippy, i'm a pion
      };\end{lstlisting}
\end{document}
